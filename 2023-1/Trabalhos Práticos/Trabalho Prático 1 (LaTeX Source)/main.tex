\documentclass{article}
\usepackage[utf8]{inputenc}
\usepackage[margin=1.2in]{geometry}
\usepackage{hyperref}

\usepackage{tikz}
  \usetikzlibrary{shapes,arrows,fit,calc,positioning}
  \tikzset{box/.style={draw, diamond, thick, text centered, minimum height=0.5cm, minimum width=1cm}}
  \tikzset{line/.style={draw, thick, -latex'}}

\usepackage{listings}
\usepackage{xcolor}

\definecolor{codegreen}{rgb}{0,0.6,0}
\definecolor{codegray}{rgb}{0.5,0.5,0.5}
\definecolor{codepurple}{rgb}{0.58,0,0.82}
\definecolor{backcolour}{rgb}{0.95,0.95,0.92}

\lstdefinestyle{mystyle}{
    backgroundcolor=\color{backcolour},   
    commentstyle=\color{codegreen},
    keywordstyle=\color{magenta},
    numberstyle=\tiny\color{codegray},
    stringstyle=\color{codepurple},
    basicstyle=\ttfamily\footnotesize,
    breakatwhitespace=false,         
    breaklines=true,                 
    captionpos=b,                    
    keepspaces=true,                 
    numbers=left,                    
    numbersep=5pt,                  
    showspaces=false,                
    showstringspaces=false,
    showtabs=false,                  
    tabsize=2
}

\lstset{style=mystyle}


\usepackage{tikz}
\usetikzlibrary{positioning}

\usepackage{natbib}
\usepackage{graphicx}
\usepackage{amsmath}

\title{\vspace{-2 cm}Python para Ciência de Dados \\ Trabalho Prático}
\author{Prof. Rodrigo Silva}
\date{}

\begin{document}

\maketitle

%\section*{Instruções}

%Cada aluno deve submeter na Plataforma Moodle um arquivo PDF com o nome no formato, \textit{seu\_nome\_intropython.pdf}, contendo:
%\begin{itemize}
%    \item Nome;
%    \item Número de Matrícula;
%    \item Repostas das questões teóricas; e
%    \item Link para o repositório do GitHub que contém o código da atividade prática. 
%\end{itemize}

\section{Leitura Recomendada }

\begin{itemize}
    \item VanderPlas, Jake. Python data science handbook: Essential tools for working with data. O'Reilly Media, Inc., 2016. Disponível em \url{https://jakevdp.github.io/PythonDataScienceHandbook/}
\end{itemize}


\section{Atividades Prática}

\begin{enumerate}
    \item  Apresentar e resolver algum aplicação de ciência de dados com as seguintes características.
    \begin{enumerate}
        \item A aplicação deve conter um problema de regressão ou de classificação.
        \item A aplicação deve envolver pelo menos duas fontes de dados.
    \end{enumerate}
\end{enumerate}

\section{Entregável}

\begin{enumerate}
    \item  O aluno deverá produzir um notebook python que apresente com os seguintes requisitos mínimos: 
    \begin{enumerate}
        \item Carregamento das fontes de dados.
        \item Fusão das fontes de dados (\texttt{merge} e/ou \texttt{join})
        \item Análise descritiva dos dados (Média, Max, Min)
        \item Análise gráfica dos dados
        \item Treinamento, ajuste fino e teste de modelos de aprendizado de máquina. 
    \end{enumerate}
    Cada uma destas etapas deve ser comentada (no próprio notebook com markdown) com uma descrição do que levou à escolha das técnicas e quais as conclusões das etapas.
\end{enumerate}

\section{Entrega}

\begin{itemize}
    \item O trabalho deve ser \textcolor{red}{finalizado até o dia 18/06}. 
    \item Os alunos devem encaminhar pelo moodle apenas o link para o notebook que deve estar disponível e público em um repositório no GitHub.
\end{itemize}
 



%\bibliographystyle{plain}
%\bibliography{references}
\end{document}

